\section{Introduction}
Recent advances in single-cell RNA sequencing provide new opportunities for studying cellular dynamic processes, such as the cell differentiation and \ldots. \cite{tanay_scalingsinglecellgenomics_2017,etzrodt_quantitativesinglecellapproaches_2014}.
Trajectory inference (TI) methods attempt to reconstruct such dynamic processes by ordering cells along a topology according to their transcriptomic similarities \cite{tanay_scalingsinglecellgenomics_2017,cannoodt_computationalmethodstrajectory_2016}. 

A recent benchmarking study of 45 TI methods found a high variability in accuracy across the different methods \cite{saelens_comparisonsinglecelltrajectory_2019}. Rather, the performance of a method on a dataset strongly depended on the type of trajectory present in the data. For instance, PAGA \cite{wolf_pagagraphabstraction_2019} performs well able to recover complex trajectories containing cycles or disconnected regions, whereas Slingshot \cite{street_slingshotcelllineage_2018} obtains high accuracy scores for trajectories containing one or several bifurcations. As part of the benchmark not only each method's accuracy was evaluated, but also other factors that affect how well a user is able to useful models for the task at hand, namely the methods scalability, stability, and usability.

Along with PAGA and Slingshot, SCORPIUS is one of the few TI methods to obtain high scores in all four of these categories. SCORPIUS is toolbox specialised in inferring, visualising and interpreting linear trajectories. Out of 69 linear datasets used in the benchmark, SCORPIUS was the only TI method capable of inferring a high-quality trajectory on more than half of those datasets. In this work, we show how to apply SCORPIUS, explain its workings, and demonstrate how each algorithmic component of the method contributed to its accuracy, scalability, and robustness.


\section{Results}

\subsection{Demonstration on dataset 1}

\subsection{Demonstration on dataset 2}

\subsection{Components}
% Sparse spearman correlation
% LMDS
% Approximated principal curves + initialisation


\section{Discussion}

\section{Methods}

\subsection{Correlation distance}

\subsection{LMDS}

\subsection{Trajectory initialisation}

\subsection{Approximated principal curves}

\subsection{Feature selection}

\subsection{Heatmap}
