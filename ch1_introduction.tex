\newpage{\thispagestyle{empty}\cleardoublepage}
\chapter{Introduction} 
\chaptermark{Introduction.} % short title
\label{chap:introduction}
\lettergroup{\thechapter}

\begin{quote}
	\textbf{Abstract:} 
	
	%Recent developments in single-cell transcriptomics have opened new opportunities for studying dynamic processes in immunology in a high-throughput and unbiased manner. Starting from a mixture of cells in different stages of a developmental process, unsupervised trajectory inference algorithms aim to automatically reconstruct the underlying developmental path that cells are following. In this review, we break down the strategies used by this novel class of methods, and organize their components into a common framework, highlighting several practical advantages and disadvantages of the individual methods. We also give an overview of new insights these methods have already provided regarding the wiring and gene regulation of cell differentiation. As the trajectory inference field is still in its infancy, we propose several future developments which will ultimately lead to a global and data-driven way of studying immune cell differentiation.
\end{quote}

\vfill

Adapted from:\\
\textbf{Cannoodt, R.}$^*$, Saelens, W.$^*$, and Saeys, Y. Computational methods for trajectory inference from single-cell transcriptomics. \textit{European Journal of Immunology} 46, 11 (2016), 2496--2506. \doi{10.1002/eji.201646347}.\\
{\footnotesize $^*$ Equal contribution}
\newpage

\section{The cell}
The cell is the smallest unit of life, of which all known living organisms are composed. Every cell houses a plethora of biomolecular processes that allow it to adapt to changes in their environment continuously. It can be very challenging to comprehend the cellular response to a signal due to the dynamic nature of these processes. A reductionist approach to understanding a complex biological system is to study the biochemical components which it is comprised of\cite{brigandt_reductionismbiology_2017}.

Reductionist biologists are delighted by recent advances in experimental technologies that permit measuring the abundance of thousands of different biochemical molecules in tens of thousands of individual cells.  Observing the biomolecular insides of cells in this manner will ultimately provide fundamental insights into the processes that govern these cells and help uncover novel approaches for diagnosing and treating disease. Every coin has its flip side, however, and in this case, it is that the amount of data generated from these experiments is not analysable by hand.  

For example, the Human Cell Atlas (HCA) consortium\cite{regev_humancellatlas_2018} has set out to develop a comprehensive reference map of all the different types of cells in the human body. Experts in the field often metaphorically describe the HCA initiative as aiming to develop a 'Google Maps' of the human body. Even in its infancy, the HCA has profiled 3.8 million cells from 248 donors across 42 labs\cite{humancellatlasconsortium_humancellatlas_2018}, and this number is likely to increase well above one hundred million.

The sheer volume of the data generated from such highly-integrative and high-throughput experiments are not the only reason why they are so challenging to interpret. Namely, the data inherently also suffers from batch effects arising from differences between donors and labs, and also contains high levels of noise arising from the experimental profiling techniques used\cite{hon_humancellatlas_2018}. Biologists thus turn to computer scientists\footnote{or computational biologists turn to themselves} to develop new tools to tackle these problems and help biologists extract meaningful biological insights from the data.

This work makes incremental contributions to the field in order to be able to address the aforementioned problems in a more comprehensive context. This chapter first introduces several fundamental concepts in both cell biology and computer science. Afterwards, the research objectives and main contributions of this work are outlined.




\subsection{Dynamic processes}

\subsection{Gene regulation}

\subsection{Single cell transcriptomics}

\section{Machine learning}

\subsection{Dimensionality reduction}

\subsection{Trajectory inference}

\subsection{Feature selection}

\subsection{Network inference}

\section{Research objectives}

\section{Outline} % include contributions?