\newpage{\thispagestyle{empty}\cleardoublepage}
\chapter{Introduction} 
\chaptermark{Introduction.} % short title
\label{chap:introduction}
\lettergroup{\thechapter}

\begin{quote}
	\textbf{Abstract:} Recent developments in single-cell transcriptomics have opened new opportunities for studying dynamic processes in immunology in a high-throughput and unbiased manner. Starting from a mixture of cells in different stages of a developmental process, unsupervised trajectory inference algorithms aim to automatically reconstruct the underlying developmental path that cells are following. In this review, we break down the strategies used by this novel class of methods, and organize their components into a common framework, highlighting several practical advantages and disadvantages of the individual methods. We also give an overview of new insights these methods have already provided regarding the wiring and gene regulation of cell differentiation. As the trajectory inference field is still in its infancy, we propose several future developments which will ultimately lead to a global and data-driven way of studying immune cell differentiation.
\end{quote}

\vfill

Adapted from:\\
\textbf{Cannoodt, R.}$^*$, Saelens, W.$^*$, and Saeys, Y. Computational methods for trajectory inference from single-cell transcriptomics. \textit{European Journal of Immunology} 46, 11 (2016), 2496--2506. \doi{10.1002/eji.201646347}.\\
{\footnotesize $^*$ Equal contribution}
\newpage

\section{Single cell biology}

\subsection{Dynamic processes}

\subsection{Gene regulation}

\subsection{Single cell transcriptomics}

\section{Machine learning}

\subsection{Dimensionality Reduction}

\subsection{Trajectory inference}

\subsection{Feature selection}

\subsection{Network inference}

\section{Research objectives}

\section{Outline} % include contributions?