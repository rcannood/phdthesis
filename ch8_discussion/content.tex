

%Before embarking on this perilous adventure, I attempt to write a "Hippocratic oath for bioinformaticians", to guide me through this dissertation and not stray from the path of righteousness.
%
%\subsection{Hippocratic oath for bioinformaticians}
%We proceed with caution in developing new tools. Its results should not only be convincing and easy to interpret, but also accurate, robust, and reproducible. We value creating open-source software that works reliably but fails gracefully when it does not. We acknowledge that writing automated tests is dull but necessary for maintaining long-term software projects. % TODO: improve

%Computer scientists should proceed with caution in developing new software, however, as the results produced should not only be convincing and easy to interpret, but the software should be robust and generate sufficiently accurate models of the underlying system.
%
%Bit too excited -- false positives, poor accuracy, scalability issues, poor software quality. 

\section{dyngen discussion}

\section{dynbenchmark discussion}
\begin{itemize}
	\item Already outdated when the manuscript was published online
	\item Update benchmark with more TI methods, newer (and larger) datasets, perform parameter optimisation on methods
	\item Include RNA velocity as inputs
\end{itemize}

\section{SCORPIUS discussion}
\begin{itemize}
	\item Extension to inferring non-linear trajectories, i.e. with principal graphs
\end{itemize}

\section{bred discussion}

\section{incgraph discussion}

\section{dyno discussion}
% notice that normalisation was absent from this dissertation.
