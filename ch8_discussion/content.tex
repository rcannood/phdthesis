% DISCUSSION CONTENT
	
This chapter reflects on the impact of this work on the field, referring back to the research objectives that were defined in Chapter \ref{chap:introduction}. 

\section{dyngen: benchmarking with \textit{in silico} single cells}
We developed a single-cell simulator that can be used to simulate types of single-cell omics data for which the technology does not yet exist (Chapter~\ref{chap:dyngen}). The \textit{in silico} single cell data can help kick-start emerging domains with low data availability more safely by making novel computational tools more reproducible. Indeed, the simulator has already been used to evaluate trajectory inference\cite{saelens_comparisonsinglecelltrajectory_2019}, trajectory alignment\cite{vandenberge_trajectorybaseddifferentialexpression_2019}, and single cell network inference\cite{pratapa_benchmarkingalgorithmsgene_2019} methods.

\section{dynbenchmark: A comparison of single-cell trajectory inference methods}
Using this simulator and a collection of real datasets, we performed a comparison of 45 TI methods (Chapter~\ref{chap:dynbenchmark}). 
Our contributions include writing software to run 45 different error-prone TI methods with a common interface, downloading and processing hundreds of single cell datasets, and developing novel metrics for comparing ground truth and predicted trajectories.

We constructed a set of guidelines for end-users to help choose a TI method that is accurate, robust and fit for their application. 
Since such guidelines were hitherto lacking, they are commonly disseminated in manuscripts \cite{lafzi_tutorialguidelinesexperimental_2018,luecken_currentbestpractices_2019}, courses \cite{kiselev_analysissinglecell_2019,martens_analysissinglecell_2019}, and slides shown during keynote caffeine refuelling sessions \cite{hemberg_coffeebreakanalysis_2019}. 

We made our pipeline, datasets, metrics, and containerised wrappers of TI methods publicly available for developers to use. Despite our best efforts, we do not observe an increase in developers performing quantitative benchmarks. We hypothesise causal reasons for this phenomenon and provide solutions in order to spur TI developers to perform more self-assessments (Chapter~\ref{chap:selfassessment}).

Over the course of four years, we developed a lot of practical experience in benchmarking complex computational tools. We summarise this experience in a set of essential guidelines for benchmarking computational tools (Chapter~\ref{chap:guidelines}).

%	\item Already outdated when the manuscript was published online
%	\item Update benchmark with more TI methods, newer (and larger) datasets, perform parameter optimisation on methods
%	\item Include RNA velocity as inputs

\section{dyno: A toolkit for inferring and interpreting trajectories}
% spinoff from dynbenchmark
% not yet published, yet a lot of activity on github
% invest time in publishing all of the subpackages on github

\section{SCORPIUS: Fast, accurate, and robust single-cell pseudotime}

%\item We introduce a novel TI method specialised in inferring linear trajectories (Chapter~\ref{chap:scorpius}). Despite linear TI being the most simple but commonly used form of trajectory inference, the benchmark demonstrated that most TI methods are not capable of producing accurate models of linear datasets.


%\section{SCORPIUS discussion}
%\begin{itemize}
%	\item Extension to inferring non-linear trajectories, i.e. with principal graphs or GNG
%\end{itemize}

\section{bred: Inferring single cell regulatory networks}

%\item We invent a new type of NI method capable of inferring the GRN of individual cells (Chapter~\ref{chap:bred}). We demonstrate this <yadeyade .. fill in when the chapter is actually written.>

\section{incgraph: Optimising regulatory networks}

%\item Every NI method has certain topological biases. We provide a tool for analysing the topological properties of large, evolving networks and use this to iteratively optimise GRN predictions (Chapter~\ref{chap:incgraph}).

\section{Supervised or unsupervised?}


\clearpage
\section{References}
\printbibliography[heading=none]
