
% * Single cell omics is expanding at rates faster than any single mind can try to conceive
%Pioneers developing revolutionary tools to process new types of data are typically faced with the problem on how to quantitatively evaluate the accuracy of their approach. After all, the amount of data required to perform a comprehensive evaluation of their tool does not yet exist and is too expensive to generate at the time. 


%In the computational jungle of single-cell omics, experts act as beacons of hope by sharing guidelines based on comprehensive benchmarking studies. Their disseminations (in the form of manuscripts \cite{lafzi_tutorialguidelinesexperimental_2018,luecken_currentbestpractices_2019}, courses \cite{kiselev_analysissinglecell_2019,martens_analysissinglecell_2019}, and slides shown during keynote caffeine refuelling sessions \cite{hemberg_coffeebreakanalysis_2019}) are crucial in leading new users, and ultimately the whole field, to better practices for performing single-cell omics analyses.

%Before embarking on this perilous adventure, I attempt to write a "Hippocratic oath for bioinformaticians", to guide me through this dissertation and not stray from the path of righteousness.
%
%\subsection{Hippocratic oath for bioinformaticians}
%We proceed with caution in developing new tools. Its results should not only be convincing and easy to interpret, but also accurate, robust, and reproducible. We open source. Our software should work reliably and fail gracefully when it does not. We acknowledge that writing automated tests is dull but necessary for maintaining long-term software projects. % TODO: improve

%Computer scientists should proceed with caution in developing new software, however, as the results produced should not only be convincing and easy to interpret, but the software should be robust and generate sufficiently accurate models of the underlying system.
%
%Bit too excited -- false positives, poor accuracy, scalability issues, poor software quality. 

\section{dyngen discussion}
% felt that upon the development of a new technologies, good quality control practices from
% other technologies were not being carried over because the data required to evaluate computational
% tools does not exist yet and is too expensive to develop.

\section{dynbenchmark discussion}
\begin{itemize}
	\item Already outdated when the manuscript was published online
	\item Update benchmark with more TI methods, newer (and larger) datasets, perform parameter optimisation on methods
	\item Include RNA velocity as inputs
\end{itemize}

\section{SCORPIUS discussion}
\begin{itemize}
	\item Extension to inferring non-linear trajectories, i.e. with principal graphs
\end{itemize}

\section{bred discussion}

\section{incgraph discussion}

\section{dyno discussion}

