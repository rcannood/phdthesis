% DISCUSSION CONTENT

\mycomment{Note: This text was written very quickly and is only meant as a temporary coat-hanger.}
	
This chapter reflects on the impact of this work on the field, referring back to the research objectives that were defined in Chapter \ref{chap:introduction}. 
	
dyngen is a succes, as we created a simulator of single cell data that is already used to evaluate trajectory\cite{saelens_comparisonsinglecelltrajectory_2019}, trajectory alignment\cite{vandenberge_trajectorybaseddifferentialexpression_2019} and single cell network inference\cite{pratapa_benchmarkingalgorithmsgene_2019}.

dynbenchmark was meant to guide users to better practices for applying trajectory inference methods, and guide developers to adopt better practices for developing accurate and robust TI methods. The first part we succeeded in, as the guidelines are commonly disseminated in manuscripts \cite{lafzi_tutorialguidelinesexperimental_2018,luecken_currentbestpractices_2019}, courses \cite{kiselev_analysissinglecell_2019,martens_analysissinglecell_2019}, and slides shown during keynote caffeine refuelling sessions \cite{hemberg_coffeebreakanalysis_2019}. However, we do not observe an increase in developers of TI methods performing quantitative benchmarks. We will discuss this in more detail in Chapter~\ref{chap:selfassessment}.

dynbenchmark was a benchmark of TI methods developed over the course of 4 years involving writing software to run 45 different error-prone TI methods with a common interface, downloading and processing hundreds of single cell datasets, developing novel metrics for comparing ground truth and predicted trajectories, writing a simulator for synthetic single cell data. Along the way, we have learned a lot about how to benchmark computational methods. We share our vision and guidelines on benchmarking computational methods in Chapter~\ref{chap:guidelines}.

Something about SCORPIUS.

Something about bred.

Something about dyno.

Something about gng?



%\section{dyngen discussion}
%% felt that upon the development of a new technologies, good quality control practices from
%% other technologies were not being carried over because the data required to evaluate computational
%% tools does not exist yet and is too expensive to develop.
%
%\section{dynbenchmark discussion}
%\begin{itemize}
%	\item Already outdated when the manuscript was published online
%	\item Update benchmark with more TI methods, newer (and larger) datasets, perform parameter optimisation on methods
%	\item Include RNA velocity as inputs
%\end{itemize}
%
%\section{SCORPIUS discussion}
%\begin{itemize}
%	\item Extension to inferring non-linear trajectories, i.e. with principal graphs
%\end{itemize}
%
%\section{bred discussion}
%
%%\section{incgraph discussion}
%
%%\section{dyno discussion}

