\appendix

%\backmatter

\newpage{\thispagestyle{empty}\cleardoublepage}
\chapter{Curriculum Vitae}
\section{Personalia}
\begin{tabular}{ll}
	Name & Robrecht Cannoodt \\
	Date of Birth & January 25th, 1990 \\
	Place of birth & Ghent \\
	Nationality & Belgian \\
	Address & Kerkstraat 75, 9070 Destelbergen \\
	Email & \href{mailto:robrecht.cannoodt@gmail.com}{robrecht.cannoodt@gmail.com} \\
	Webpage & \url{https://www.cannoodt.dev}
\end{tabular}


\section{Professional Experience}
\textit{Ph.D. student in Computer Science} \\
\hspace*{2 ex} Ghent University, Belgium, September 2013 -- now \\
\textit{Laboratory assistant (summer job)} \\
\hspace*{2 ex} Bloedtransfusiecentrum Gent, Belgium, July (2009, 2010, 2011, 2012)


\section{Education} 
\textit{Master of Science in Computer Science Engineering: Software Engineering} \\
\hspace*{2 ex} Ghent University, Belgium, 2011 -- 2013 \vspace*{1mm}\\ 
\textit{Bachelors degree in Informatics}\\
\hspace*{2 ex} Ghent University, Belgium, 2008 -- 2011 \vspace*{1mm} \\
\textit{Bachelors degree in Engineering: Architecture} \\
\hspace*{2 ex} Ghent University, Belgium, 2007 -- 2008  \vspace*{1mm}\\
\textit{International Baccalaureate} \\
\hspace*{2 ex} International School of Berne, Switzerland, 2002 -- 2007


\section{First-author publications}

\begin{itemize}
	\item \textbf{Cannoodt R}, Saelens W, Sichien D, Tavernier S, Janssens S, Guilliams M, Lambrecht B, De Preter K, Saeys Y. SCORPIUS improves trajectory inference and identifies novel modules in dendritic cell development. bioRxiv 079509. 2016 Oct.
	\item \textbf{Cannoodt R} *, Saelens W *, Saeys Y. Computational methods for trajectory inference from single-cell transcriptomics. European journal of immunology. 2016 Nov;46(11):2496-506.
	\item \textbf{Cannoodt R}, Ruyssinck J, Ramon J, De Preter K, Saeys Y. IncGraph: Incremental graphlet counting for topology optimisation. PloS one. 2018 Apr 26;13(4):e0195997.
	\item Saelens W *, \textbf{Cannoodt R} *, Todorov H, Saeys Y. A comparison of single-cell trajectory inference methods. Nature biotechnology. 2019 May;37(5):547.
\end{itemize}

*: Equal contribution.

\section{Co-author publications}
\begin{itemize}
	\item Decock A, Ongenaert M, \textbf{Cannoodt R}, Verniers K, De Wilde B, Laureys G, Van Roy N, Berbegall AP, Bienertova-Vasku J, Bown N, Clément N. Methyl-CpG-binding domain sequencing reveals a prognostic methylation signature in neuroblastoma. Oncotarget. 2016 Jan 12;7(2):1960.
	\item Van Cauwenbergh C, Van Schil K, \textbf{Cannoodt R}, Bauwens M, Van Laethem T, De Jaegere S, Steyaert W, Sante T, Menten B, Leroy BP, Coppieters F. arrEYE: a customized platform for high-resolution copy number analysis of coding and noncoding regions of known and candidate retinal dystrophy genes and retinal noncoding RNAs. Genetics in Medicine. 2017 Apr;19(4):457.
	\item Claeys S, Denecker G, \textbf{Cannoodt R}, Kumps C, Durinck K, Speleman F, De Preter K. Early and late effects of pharmacological ALK inhibition on the neuroblastoma transcriptome. Oncotarget. 2017 Dec 5;8(63):106820.
	\item Depuydt P, Boeva V, Hocking TD, \textbf{Cannoodt R}, Ambros IM, Ambros PF, Asgharzadeh S, Attiyeh EF, Combaret V, Defferrari R, Fischer M. Genomic amplifications and distal 6q loss: novel markers for poor survival in high-risk neuroblastoma patients. JNCI: Journal of the National Cancer Institute. 2018 Mar 5;110(10):1084-93.
	\item Scott CL*, T'Jonck W*, Martens L, Todorov H, Sichien D, Soen B, Bonnardel J, De Prijck S, Vandamme N, \textbf{Cannoodt R}, Saelens W, Vanneste B, Toussaint W, De Bleser P, Takahashi N, Vandenabeele P, Henri S, Pridans C, Hume DA, Lambrecht BN, De Baetselier P, Milling SWF, Van Ginderachter JA, Malissen B, Berx G, Beschin A, Saeys Y, Guilliams M. The transcription factor ZEB2 is required to maintain the tissue-specific identities of macrophages. Immunity. 2018 Aug 21;49(2):312-25.
	\item Saelens W, \textbf{Cannoodt R}, Saeys Y. A comprehensive evaluation of module detection methods for gene expression data. Nature communications. 2018 Mar 15;9(1):1090.
	\item Todorov H, \textbf{Cannoodt R}, Saelens W, Saeys Y. Network Inference from Single-Cell Transcriptomic Data. In Gene Regulatory Networks 2019 (pp. 235-249). Humana Press, New York, NY.
	\item Van den Berge K, De Bezieux HR, Street K, Saelens W, \textbf{Cannoodt R}, Saeys Y, Dudoit S, Clement L. Trajectory-based differential expression analysis for single-cell sequencing data. bioRxiv. 2019 Jan 1:623397.
	\item Weber LM, Saelens W, \textbf{Cannoodt R}, Soneson C, Hapfelmeier A, Gardner PP, Boulesteix AL, Saeys Y, Robinson MD. Essential guidelines for computational method benchmarking. Genome biology. 2019 Jun;20(1):125.
	\item Lorenzi L*, Hua-Sheng C*, Avila Cobos F, Gross S, Volders PJ, \textbf{Cannoodt R}, Nuytens J, Vanderheyden K, Anckaert J, Lefever S, Goovaerts T, Hansen TB, Kuersten S, Nijs N, Taghon T, Vermaelen K, Brache KR, Saeys Y, De Meyer T, Deshpande N, Anande G, Chen TW, Wilkins MR, Unnikrishnan A, De Preter K, Kjerns J, Koster J, Schroth GP, Vandesompele J, Surnazin P, Mestdagh P. The RNA Atlas, a single nucleotide resolution map of the human transcriptome. bioRxiv. 2019 Oct:807529. Submitted to Nature. 
	\item Van den Berge K, Roux de Bézieux H, Street K, Saelens W, \textbf{Cannoodt R}, Saeys Y, Dudoit S. Trajectory-based differential expression analysis. Submitted to Nature Communications.
	\item Van de Sande B, Flerin C, Davie K, De Waegeneer M, Hulselmans G, Aibar S, Seurinck R, Saelens W, \textbf{Cannoodt R}, Rouchon Q, Verbeiren T, De Maeyer D, Reumers J, Saeys Y, Aerts S. A scalable SCENIC workflow for single-cell gene regulatory network analysis. Submitted to Nature Protocols.
\end{itemize}

*: Equal contribution.

\section{Conferences and meetings}

\begin{itemize}
  \item \textbf{Differential Network Medicine}, Antwerp, Belgium, 4-6 December 2013.
  \item \textbf{BeNeLux Bioinformatics Conference}, Brussels, Belgium, 9-10 December 2013.
  \item \textbf{OncoPoint}, Ghent, Belgium, 6 February 2014.
  \item \textbf{AISTATS}, Reykjavik, Iceland, 22-25 April 2014.
  \item \textbf{N2N Annual Symposium}, Ghent, Belgium, 21 May 2014.
  \item \textbf{Bioinformatics N2N Conference}, Ghent, Belgium, 21 May 2014.
  \item \textbf{BENELEARN}, Brussels, Belgium, 6 June 2014.
  \item \textbf{BeNeLux Bioinformatics Conference}, Luxembourg, 8-9 December 2014.
  \item \textbf{OncoPoint}, Ghent, Belgium, 11 February 2015.
  \item \textbf{Big Data to Bedside}, Ghent, Belgium, 1-2 April 2015.
  \item \textbf{BIG N2N symposium}, Ghent, Belgium, 21 May 2015.
  \item \textbf{BeNeLux Bioinformatics Conference}, Antwerp, Belgium, 7-8 December 2015
  \item \textbf{Single Cell Biology Workshop}, Ghent, Belgium, 15 January 2016.
  \item \textbf{Single Cell Biology}, Hinxton, United Kingdom, 8-10 March 2016.
  \item \textbf{BIG N2N symposium}, Ghent, Belgium, 19 May 2016.
  \item \textbf{CYTO}, Seattle, USA, 11-15 June 2016.
  \item \textbf{BENELEARN}, Kortrijk, Belgium, 12-13 September 2016.
  \item \textbf{Single Cell Genomics}, Hinxton, United Kingdom, 14-16 September 2016.
  \item \textbf{Single Cell Biology Workshop}, Ghent, Belgium, 15 January 2016.
  \item \textbf{VIB Seminar}, Veldhoven, Netherlands, 27-28 April 2017.
  \item \textbf{Keystone Symposia: Single Cell Omics}, Stockholm, Sweden, 26-30 May 2017. 
  \item \textbf{BeNeLux Bioinformatics Conference}, Louvain, Belgium, 13-14 December 2017.
  \item \textbf{Single Cell Biology}, Hinxton, United Kingdom, 8-10 March 2018.
  \item \textbf{Keystone Symposia: Single Cell Biology}, Colorado, USA, 13-17 January 2019.
  \item \textbf{Human Cell Atlas}, Toronto, Canada, 29-31 July 2019.
\end{itemize}

\section{Courses / workshops}

\begin{itemize}
  \item \textbf{Differential Network Medicine}, Antwerp, Belgium, 4-6 September 2013.
  \item \textbf{Basics of Biology for Engineers}, Louvain, Belgium, 18-20 September 2013.
  \item \textbf{Metric Learning}, Louvain, Belgium, 15 October 2013.
  \item \textbf{Machine learning Summer School}, Reykjavik, Iceland, 25 April - 4 May 2014.
  \item \textbf{Effective Oral Presentations by Jean-Luc Doumont}, Ghent, Belgium, 3 February 2015.
  \item \textbf{RNA-Seq analysis for differential expression}, Ghent, Belgium, 24-27 April 2015.
  \item \textbf{BigData@UGent in practice}, Ghent, Belgium, 4 May 2015.
\end{itemize}

\section{Oral presentations}

\begin{itemize}
  \item Cannoodt R., Ruyssinck J., De Preter K., Dhaene T., Saeys Y. : Network inference by integrating biclustering and feature selection. \textbf{BeNeLux Bioinformatics Conference}, Brussels, Belgium, 9-10 December 2013.
  \item Cannoodt R., Ruyssinck J., De Preter K., Dhaene T., Saeys Y. : Network inference by integrating biclustering and feature selection. \textbf{IRC bioinformatics day}, Ghent, Belgium, 23 November 2013.
  \item Cannoodt R., De Preter K., Saeys Y. : Differential network inference for pediatric cancers. \textbf{Maestra meeting}, Ghent, Belgium, 14 March 2014.
  \item Cannoodt R., Beckers A., Van Cauwenbergh C., Speleman F., Saeys Y., De Preter K. : Differential module analysis in neuroblastoma regulatory networks. \textbf{Oncopoint}, Ghent, Belgium, 11 February 2015.
  \item Cannoodt R., Saelens W., De Preter K., Saeys Y.: Inferring developmental chronologies from single cell RNA, \textbf{BeNeLux Bioinformatics Conference}, Antwerp, Belgium, 7-8 December 2015.
  \item Cannoodt R., Saelens W., De Preter K., Saeys Y.: Inferring trajectories along dynamic processes from single-cell RNA-seq data. \textbf{Jean-Luc Doumont workshop}, Ghent, Belgium, 17 December 2015.
  \item Cannoodt R., Saelens W., De Preter K., Saeys Y. : SCORPIUS: Inferring trajectories along dynamic processes from single-cell RNA-seq data. \textbf{Single Cell Biology Workshop}, Ghent, Belgium, 15 January 2016.
  \item Cannoodt R., Saelens W., De Preter K., Saeys Y. : Improving marker gene discovery from high-dimensional single-cell snapshot data. \textbf{CYTO}, Seattle, USA, 11-15 June 2016.
  \item Cannoodt R., Saelens W., De Preter K., Saeys Y. : Unbiased modelling of dynamic processes with single-cell RNA-sequencing. \textbf{BENELEARN}, Kortrijk, Belgium, 12-13 September 2016.
  \item Cannoodt R., Saelens W., Sichien D., Tavernier S., Janssens S., Guilliams M., Lambrecht B., De Preter K., Saeys Y. : Unbiased modelling of dynamic processes with SCORPIUS identifies novel modules in dendritic cell development. \textbf{VIB Seminar}, Veldhoven, Netherlands, 27-28 april 2017.
  \item Cannoodt R., Saelens W. : Automated building and unit testing, Docker and Singularity. \textbf{VIB Developers Meeting}, Ghent, Belgium, 25 January 2019.
  \item Cannoodt R., Saelens W., Todorov H., Saeys Y. : dynbenchmark, Assessing Accuracy, Robustness and Usability of Single-Cell Trajectory Inference methods. \textbf{Keystone Symposia: Single Cell Biology}, Colorado, USA, 13-17 January 2019.
\end{itemize}

\section{Poster presentations}

\begin{itemize}
  \item Cannoodt R., Ruyssinck J., De Preter K., Dhaene T., Saeys Y.: Network Inference by Integrating Biclustering and Feature Selection. \textbf{N2N Annual Symposium}, Ghent, Belgium, 21 May 2014.
  \item Cannoodt R., Van Cauwenbergh C., Beckers A., Speleman F., De Preter K., Saeys Y.: Differential Module Analysis in Neuroblastoma Regulatory Networks. \textbf{BeNeLux Bioinformatics Conference} 2014, Luxembourg, Belgium, 8-9 December 2014.
  \item Cannoodt R., Beckers A., Van Cauwenbergh C., Speleman F., De Preter K., Saeys Y.: Differential Module Analysis in Neuroblastoma Regulatory Networks, \textbf{BIG N2N symposium}, Ghent, Belgium, 21 May 2015.
  \item Cannoodt R., Saelens W., De Preter K., Saeys Y. : Unbiased modelling of dynamic processes with single-cell RNA-sequencing. Single Cell Biology, Hinxton, United Kingdom, 8-10 March 2016.
  \item Cannoodt R., Saelens W., De Preter K., Saeys Y. : Unbiased modelling of dynamic processes with single-cell RNA-sequencing. \textbf{BIG N2N symposium}, Ghent, Belgium, 19 May 2016.
  \item Cannoodt R., Saelens W., De Preter K., Saeys Y. : Improving marker gene discovery from high-dimensional single-cell snapshot data. CYTO, Seattle, USA, 11-15 June 2016.
  \item Cannoodt R., Saelens W., De Preter K., Saeys Y. : Unbiased modelling of dynamic processes with single-cell RNA-sequencing. BENELEARN 2016, Kortrijk, Belgium, 12-13 September 2016.
  \item Cannoodt R., Saelens W., De Preter K., Saeys Y. : Unbiased modelling of dynamic processes with single-cell RNA-sequencing. Single Cell Genomics, Hinxton, United Kingdom, 14-16 September 2016.
  \item Cannoodt R., Saelens W., Sichien D., Tavernier S., Janssens S., Guilliams M., Lambrecht B., De Preter K., Saeys Y.: Unbiased modelling of dynamic processes with SCORPIUS identifies novel modules in dendritic cell development. \textbf{VIB Seminar}, Veldhoven, Netherlands, 27-28 April 2017.
  \item Cannoodt R., Saelens W., De Preter K., Saeys Y.: True single cell network inference: Modelling gene regulation of individual cells. \textbf{Keystone Symposia: Single Cell Omics}, Stockholm, Sweden, 26-30 May 2017. 
  \item Cannoodt R., Saelens W., Todorov H., Saeys Y.: Generalised framework for and comparison of 24 trajectory inference methods. \textbf{BeNeLux Bioinformatics Conference}, Louvain, Belgium, 13-14 December 2017.
  \item Cannoodt R., Saelens W., Todorov H., Saeys Y. : A comparison of single-cell trajectory inference methods: towards more accurate and robust tools. \textbf{Single Cell Biology}, Hinxton, United Kingdom, 8-10 March 2018.
  \item Cannoodt R., Saelens W. : dynbenchmark, Assessing Accuracy, Robustness and Usability of Single-Cell Trajectory Inference methods. \textbf{Keystone Symposia: Single Cell Biology}, Colorado, USA, 13-17 January 2019.
\end{itemize}

\section{Master student supervision}

\begin{itemize}
  \item \textbf{Leen De Baets}. Identificatie van nieuwe kankergenen voor neuroblastoomonderzoek met machine learning. September 2013 -- June 2014.
  \item \textbf{Wouter Saelens}. Locale cel-type specifieke genexpressie in het myeloïde transcriptoom. September 2013 -- June 2014.
  \item \textbf{Charlotte De Vogelaere}. Quantitative evaluation of network inference methods for single-cell cancer regulomes. September 2015 -- June 2016.
  \item \textbf{Sofie Veys}. Comparative review of dimensionality reduction methods for high-throughput single-cell transcriptomics. September 2016 -- June 2017.
  \item \textbf{Chloë Guidi}. Het afleiden van dynamische grafen op basis van snapshot data. September 2016 -- June 2017.
  \item \textbf{Jarre Knockaert}. Inferentie van cellontwikkelingstrajecten met machine learning. September 2018 -- June 2019.
\end{itemize}

\section{Open-source software}
As part of this work, many open-source software packages were created and many others were contributed to (Table \ref{tab:packages}). 

Packages that were created as part of this work are hosted on Github under the username rcannood\footnote{\url{https://github.com/rcannood?tab=repositories}} or the dynverse organisation\footnote{\url{https://github.com/dynverse?tab=repositories}}. As part of our standard development practices, we automate execution of unit tests and write extensive documentation to ensure the code complies with CRAN policy before submission. Many of the packages are already hosted on CRAN, or are in the process of being prepared for submission.

We also helped maintain or extend other packages on Github, CRAN or Bioconductor on which our software depends. This includes speeding up parts of the dependency (slingshot), implementing new functionality (devtools, ParamHelpers, ranger, rlang), fixing bugs (proxyC, rlang, monocle, splatter, slingshot), becoming a maintainer of orphaned packages (diffusionMap, princurve, GillespieSSA), and extending the documentation (devtools, mlr, remotes, tidyverse). Several of these packages are mainstream R packages which receive millions of downloads per year (devtools, ranger, remotes, rlang, tidyverse).

\newcommand{\cranpkg}[1]{\href{https://cran.r-project.org/package=#1}{#1}}
\newcommand{\biocpkg}[1]{\href{https://bioconductor.org/packages/#1}{#1}}
\newcommand{\biocpkgl}[2]{\href{https://bioconductor.org/packages/#1}{#2}}
\newcommand{\githubpkg}[2]{\href{https://github.com/#1/#2}{#2}}
\newcommand{\notavailable}{}

\begin{table}[ht!]
	\caption{\textbf{Contibutions to open-source software.} Following abbreviations denote the relation with respect to the package: \textit{aut} Author, \textit{ctb} Contributor. Yearly download statistics are based on the number of downloads between 2019-10-01 and 2019-11-28. CRAN download statistics are retrieved from the Rstudio CRAN mirror only; other CRAN mirrors do not track download statistics. In addition, many of the dynverse packages have only recently been published on CRAN. For Github repositories, no download statistics could be retrieved. } \label{tab:packages}
	
	\centering\fontsize{7}{9}\selectfont
	\begin{tabularx}{\linewidth}{|p{2cm}llp{1.5cm}X|}
		\hline
		Name & Role & Host & Downloads per year & Description \\ \hline\hline
		\cranpkg{babelwhale} & aut & CRAN & 6110 & Interacting with Docker and Singularity containers \\
		\cranpkg{diffusionMap} & aut & CRAN & 30'123 & Implements diffusion map method of data parameterization \\
		\githubpkg{dynverse}{dynbenchmark} & aut & Github & \notavailable & Pipeline for benchmarking trajectory inference methods \\
		\cranpkg{dyndimred} & aut & CRAN & 5116 & Applying dimensionality reduction methods \\
		\githubpkg{dynverse}{dyneval} & aut & Github & \notavailable & Evaluating trajectory inference methods \\
		\githubpkg{dynverse}{dynfeature} & aut & Github & \notavailable & Calculating feature importance scores from trajectories \\
		\githubpkg{dynverse}{dyngen} & aut & Github & \notavailable & Simulating single-cell data using gene regulatory networks \\
		\githubpkg{dynverse}{dynguidelines} & aut & Github & \notavailable & User guidelines for trajectory inference \\
		\githubpkg{dynverse}{dynmethods} & aut & Github & \notavailable & A collection of wrappers for trajectory inference methods \\
		\githubpkg{dynverse}{dyno} & aut & Github & \notavailable & A pipeline for inferring, visualising and interpreting trajectories \\
		\cranpkg{dynparam} & aut & CRAN & 3816 & Creating meta-information for parameters \\
		\githubpkg{dynverse}{dynplot} & aut & Github & \notavailable & A simple visualisation library for trajectories \\
		\githubpkg{dynverse}{dynplot2} & aut & Github & \notavailable & A fully customisable visualisation library for trajectories \\
		\githubpkg{dynverse}{dyntoy} & aut & Github & \notavailable & Generating simple toy data of cellular differentiation \\
		\cranpkg{dynutils} & aut & CRAN & 16'999 & Common functionality for the dynverse packages \\
		\cranpkg{dynwrap} & aut & CRAN & 4009 & A common format for trajectories \\
		\cranpkg{GillespieSSA} & aut & CRAN & 7763 & Gillespie's Stochastic Simulation Algorithm (SSA) \\
		\cranpkg{GillespieSSA2} & aut & CRAN & 4181 & Gillespie's Stochastic Simulation Algorithm for Impatient People \\
		\githubpkg{dynverse}{gng} & aut & Github & \notavailable & Growing Neural Gas implemented in Rcpp \\
		\cranpkg{incgraph} & aut & CRAN & 3570 & Incremental graphlet counting for network optimisation \\
		\cranpkg{lmds} & aut & CRAN & 1742 & Landmark Multi-Dimensional Scaling \\
		\cranpkg{princurve} & aut & CRAN & 28'869 & Fits a principal curve in arbitrary dimension \\
		\cranpkg{proxyC} & aut & CRAN & 122'858 & Computes proximity in large sparse matrices \\
		\cranpkg{qsub} & aut & CRAN & 3622 & Running commands remotely on gridengine clusters \\
		\cranpkg{SCORPIUS} & aut & CRAN & 4285 & Inferring developmental chronologies from single-cell RNA sequencing data \\ \hline\hline
		
		\cranpkg{badger} & ctb & CRAN & 6472 & Query information and generate badge for using in README \\
		\biocpkgl{ClusterSignificance}{Clus\-ter\-Sig\-nif\-i\-cance} &  & Bioc & 935 & Assess if class clusters in dimensionality reduced data representations have a separation different from permuted data \\
		\cranpkg{devtools} & ctb & CRAN & 5'918'700 & Tools to make developing R packages easier \\
		\cranpkg{ggrepel} & ctb & CRAN & 2'018'030 & Repel overlapping text labels away from each other \\
		\githubpkg{soedinglab}{merlot} & ctb & Github & \notavailable & Reconstructing lineage-tree topologies from scRNA-seq data \\
		\cranpkg{mlr} & ctb & CRAN & 176'330 & Machine Learning in R \\
		\biocpkg{monocle} & ctb & Bioc & 34'360 & Clustering, differential expression, and trajectory analysis for single-cell RNA-Seq \\
		\cranpkg{ParamHelpers} & ctb & CRAN & 150'775 & Helpers for Parameters in Black-Box Optimization, Tuning and Machine Learning \\
		\githubpkg{kieranrcampbell}{pseudogp} & ctb & Github & \notavailable & Probabilistic pseudotime for single-cell RNA-seq \\
		\cranpkg{ranger} & ctb & CRAN & 413'641 & A Fast Implementation of Random Forests \\
		\cranpkg{Rdimtools} & ctb & CRAN & 7367 & Dimension Reduction and Estimation Methods \\
		\cranpkg{remotes} & ctb & CRAN & 3'944'090 & R package installation from remote repositories \\
		\cranpkg{rlang} & ctb & CRAN & 13'269'115 & Functions for base types and core R and tidyverse features \\
		\githubpkg{aertslab}{SCope} & ctb & Github & \notavailable & Visualization of high dimensional single cell data \\
		\cranpkg{shadowtext} & ctb & CRAN & 6822 & shadow text for grid and ggplot2 \\
		\biocpkg{slingshot} & ctb & Bioc & 12'085 & Tools for ordering single-cell sequencing \\
		\biocpkg{splatter} & ctb & Bioc & 5015 & Simple simulation of single-cell RNA sequencing data \\
		\cranpkg{tidyverse} & ctb & CRAN & 5'079'398 & Easily install and load packages from the tidyverse \\
		\githubpkg{farrelja}{URD} & ctb & Github & \notavailable & Reconstructing branching trajectories from single-cell RNAseq data \\
		\githubpkg{ManuSetty}{wishbone} & ctb & Github & \notavailable & Identify bifurcating developmental trajectories from single-cell data \\\hline
	\end{tabularx}
\end{table}


\section{Sources of funding}
Robrecht Cannoodt was supported by the Fonds Wetenschappelijk Onderzoek (11Y6218N).

% TODO: copyright

\newpage{\thispagestyle{empty}\cleardoublepage}
\chapter*{Acknowledgements}

This dissertation would never have seen the light of day were it not for the continued support of many friends, family, and colleagues. While writing the dissertation, Caro told me that the right place to thank everybody was at the end of the book. It was not specified \textit{which} end of the book I should be using for this. This isn't the Acknowledgement section you're looking for. 


